\documentclass[12pt,a4paper]{book}

\usepackage[a4paper,text={16.5cm,25.2cm},centering]{geometry}
\usepackage{lmodern}
\usepackage{amssymb,amsmath}
\usepackage{bm}
\usepackage{graphicx}
\usepackage{microtype}
\usepackage{hyperref}
\usepackage{amsthm}
\setlength{\parindent}{0pt}
\setlength{\parskip}{1.2ex}
\let\QED=\blacksquare
\def\bbD{{\mathbb D}}
\def\ldq{``}

\hypersetup
       {   pdfauthor = { {{Sheehan Olver}} },
           pdftitle={ {{MATH50003 Numerical Analysis}} },
           colorlinks=TRUE,
           linkcolor=black,
           citecolor=blue,
           urlcolor=blue
       }

\title{ MATH50003 Numerical Analysis }


\newtheorem{lemma}{Lemma}
\newtheorem{theorem}{Theorem}
\newtheorem{proposition}{Proposition}

\theoremstyle{definition}
\newtheorem{definition}{Definition}
\newtheorem{example}{Example}

\author{ Sheehan Olver }
\renewcommand{\thechapter}{\Roman{chapter}}


\def\addtab#1={#1\;&=}

\def\meeq#1{\def\ccr{\\\addtab}
%\tabskip=\@centering
 \begin{align*}
 \addtab#1
 \end{align*}
  }  
  
  \def\leqaddtab#1\leq{#1\;&\leq}
  \def\mleeq#1{\def\ccr{\\\addtab}
%\tabskip=\@centering
 \begin{align*}
 \leqaddtab#1
 \end{align*}
  }  


\def\vc#1{\mbox{\boldmath$#1$\unboldmath}}

\def\vcsmall#1{\mbox{\boldmath$\scriptstyle #1$\unboldmath}}

\def\vczero{{\mathbf 0}}


%\def\beginlist{\begin{itemize}}
%
%\def\endlist{\end{itemize}}


\def\pr(#1){\left({#1}\right)}
\def\br[#1]{\left[{#1}\right]}
\def\fbr[#1]{\!\left[{#1}\right]}
\def\set#1{\left\{{#1}\right\}}
\def\ip<#1>{\left\langle{#1}\right\rangle}
\def\iip<#1>{\left\langle\!\langle{#1}\right\rangle\!\rangle}

\def\norm#1{\left\| #1 \right\|}

\def\abs#1{\left|{#1}\right|}
\def\fpr(#1){\!\pr({#1})}

\def\Re{{\rm Re}\,}
\def\Im{{\rm Im}\,}

\def\floor#1{\left\lfloor#1\right\rfloor}
\def\ceil#1{\left\lceil#1\right\rceil}


\def\mapengine#1,#2.{\mapfunction{#1}\ifx\void#2\else\mapengine #2.\fi }

\def\map[#1]{\mapengine #1,\void.}

\def\mapenginesep_#1#2,#3.{\mapfunction{#2}\ifx\void#3\else#1\mapengine #3.\fi }

\def\mapsep_#1[#2]{\mapenginesep_{#1}#2,\void.}


\def\vcbr{\br}


\def\bvect[#1,#2]{
{
\def\dots{\cdots}
\def\mapfunction##1{\ | \  ##1}
\begin{pmatrix}
		 \,#1\map[#2]\,
\end{pmatrix}
}
}

\def\vect[#1]{
{\def\dots{\ldots}
	\vcbr[{#1}]
}}

\def\vectt[#1]{
{\def\dots{\ldots}
	\vect[{#1}]^{\top}
}}

\def\Vectt[#1]{
{
\def\mapfunction##1{##1 \cr} 
\def\dots{\vdots}
	\begin{pmatrix}
		\map[#1]
	\end{pmatrix}
}}



\def\thetaB{\mbox{\boldmath$\theta$}}
\def\zetaB{\mbox{\boldmath$\zeta$}}


\def\newterm#1{{\it #1}\index{#1}}


\def\TT{{\mathbb T}}
\def\C{{\mathbb C}}
\def\R{{\mathbb R}}
\def\II{{\mathbb I}}
\def\F{{\mathcal F}}
\def\E{{\rm e}}
\def\I{{\rm i}}
\def\D{{\rm d}}
\def\dx{\D x}
\def\ds{\D s}
\def\dt{\D t}
\def\CC{{\cal C}}
\def\DD{{\cal D}}
\def\U{{\mathbb U}}
\def\A{{\cal A}}
\def\K{{\cal K}}
\def\DTU{{\cal D}_{{\rm T} \rightarrow {\rm U}}}
\def\LL{{\cal L}}
\def\B{{\cal B}}
\def\T{{\cal T}}
\def\W{{\cal W}}


\def\tF_#1{{\tt F}_{#1}}
\def\Fm{\tF_m}
\def\Fab{\tF_{\alpha,\beta}}
\def\FC{\T}
\def\FCpmz{\FC^{\pm {\rm z}}}
\def\FCz{\FC^{\rm z}}

\def\tFC_#1{{\tt T}_{#1}}
\def\FCn{\tFC_n}

\def\rmz{{\rm z}}

\def\chapref#1{Chapter~\ref{Chapter:#1}}
\def\secref#1{Section~\ref{Section:#1}}
\def\exref#1{Exercise~\ref{Exercise:#1}}
\def\lmref#1{Lemma~\ref{Lemma:#1}}
\def\propref#1{Proposition~\ref{Proposition:#1}}
\def\warnref#1{Warning~\ref{Warning:#1}}
\def\thref#1{Theorem~\ref{Theorem:#1}}
\def\defref#1{Definition~\ref{Definition:#1}}
\def\probref#1{Problem~\ref{Problem:#1}}
\def\corref#1{Corollary~\ref{Corollary:#1}}

\def\sgn{{\rm sgn}\,}
\def\Ai{{\rm Ai}\,}
\def\Bi{{\rm Bi}\,}
\def\wind{{\rm wind}\,}
\def\erf{{\rm erf}\,}
\def\erfc{{\rm erfc}\,}
\def\qqquad{\qquad\quad}
\def\qqqquad{\qquad\qquad}


\def\spand{\hbox{ and }}
\def\spodd{\hbox{ odd}}
\def\speven{\hbox{ even}}
\def\qand{\quad\hbox{and}\quad}
\def\qqand{\qquad\hbox{and}\qquad}
\def\qfor{\quad\hbox{for}\quad}
\def\qqfor{\qquad\hbox{for}\qquad}
\def\qas{\quad\hbox{as}\quad}
\def\qqas{\qquad\hbox{as}\qquad}
\def\qor{\quad\hbox{or}\quad}
\def\qqor{\qquad\hbox{or}\qquad}
\def\qqwhere{\qquad\hbox{where}\qquad}



%%% Words

\def\naive{na\"\i ve\xspace}
\def\Jmap{Joukowsky map\xspace}
\def\Mobius{M\"obius\xspace}
\def\Holder{H\"older\xspace}
\def\Mathematica{{\sc Mathematica}\xspace}
\def\apriori{apriori\xspace}
\def\WHf{Weiner--Hopf factorization\xspace}
\def\WHfs{Weiner--Hopf factorizations\xspace}

\def\Jup{J_\uparrow^{-1}}
\def\Jdown{J_\downarrow^{-1}}
\def\Jin{J_+^{-1}}
\def\Jout{J_-^{-1}}



\def\bD{\D\!\!\!^-}




\def\questionequals{= \!\!\!\!\!\!{\scriptstyle ? \atop }\,\,\,}

\def\elll#1{\ell^{\lambda,#1}}
\def\elllp{\ell^{\lambda,p}}
\def\elllRp{\ell^{(\lambda,R),p}}


\def\elllRpz_#1{\ell_{#1{\rm z}}^{(\lambda,R),p}}


\def\sopmatrix#1{\begin{pmatrix}#1\end{pmatrix}}


\def\bbR{{\mathbb R}}
\def\bbC{{\mathbb C}}


\begin{document}

\maketitle

\tableofcontents

\chapter{Calculus on a Computer}

In this first chapter we explore the basics of mathematical computing and numerical analysis.
In particular we investigate the following mathematical problems which can not in general be solved exactly:

\begin{enumerate}
\item Integration. General integrals have no closed form expressions. Can we use a computer to approximate the values of definite integrals?
\item Differentiation. Differentiating a formula as in calculus is usually algorithmic, however, it is often needed to compute derivatives without access to an underlying formula, eg,  a function defined only in code. Can we use a computer to approximate derivatives?  A very important application is in Machine Learning, where there is a need to compute gradients to determine the ``right" weights in a neural network. 
\item Root finding. There is no general formula for finding roots (zeros) of arbitrary functions, or even polynomials that are of degree 5 (quintics) or higher. Can we compute roots of general functions using a computer?
\end{enumerate}

In this chapter we discuss:

\begin{enumerate}
\item I.1 Rectangular rule: We review the rectangular rule for integration and deduce the {\it converge rate} of the approximation. In the lab/problem sheet  we investigate its implementation as well as extensions to the Trapezium rule. 
\item I.2 Divided differences: We investigate approximating derivatives by a divided difference and again deduce the convergence rates. In the lab/problem sheet we extend the approach to the central differences formula and computing second derivatives. We also observe a mystery: the approximations may have significant errors in practice, and there is a limit to the accuracy.
\item I.3 Dual numbers:  We introduce the algebraic notion of a {\it dual number} which allows the implemention of {\it forward-mode automatic differentiation}, a high accuracy alternative to divided differences for computing derivatives.
\end{enumerate}




\section{Rectangular rule}
One possible definition for an integral is the limit of a Riemann sum, for example:
\[
  \ensuremath{\int}_a^b f(x) {\rm d}x = \lim_{n \ensuremath{\rightarrow} \ensuremath{\infty}} h \ensuremath{\sum}_{j=1}^n f(x_j)
\]
where $x_j = a+jh$ are evenly spaced points dividing up the interval $[a,b]$, that is  with the \emph{step size} $h = (b-a)/n$. This suggests an algorithm known as the \emph{(right-sided) rectangular rule} for approximating an integral: choose $n$ large so that
\[
  \ensuremath{\int}_a^b f(x) {\rm d}x \ensuremath{\approx} h \ensuremath{\sum}_{j=1}^n f(x_j).
\]
In the lab we explore practical implementation of this approximation, and observe that the error in approximation is bounded by $C/n$ for some constant $C$. This can be expressed using \ensuremath{\ldq}Big-O" notation:
\[
\ensuremath{\int}_a^b f(x) {\rm d}x = h \ensuremath{\sum}_{j=1}^n f(x_j) + O(1/n).
\]
In these notes we consider the \ensuremath{\ldq}Analysis" part of \ensuremath{\ldq}Numerical Analysis": we want to \emph{prove} the convergence rate of the approximation, including finding an explicit expression for the constant $C$.

To tackle this question we consider the error incurred on a single panel $(x_{j-1},x_j)$, then sum up the errors on rectangles.

Now for a secret. There are only so many tools available in analysis (especially at this stage of your career), and one can make a safe bet that the right tool in any analysis proof is either (1) integration-by-parts, (2) geometric series or (3) Taylor series. In this case we use (1):

\begin{lemma}[(Right-sided) Rectangular Rule error on one panel] Assuming $f$ is differentiable we have
\[
\ensuremath{\int}_a^b f(x) {\rm d}x = (b-a) f(b) + \ensuremath{\delta}
\]
where $|\ensuremath{\delta}| \ensuremath{\leq} M (b-a)^2$ for $M = \sup_{a \ensuremath{\leq} x \ensuremath{\leq} b}|f'(x)|$.

\end{lemma}
\textbf{Proof} We write
\meeq{
\ensuremath{\int}_a^b f(x) {\rm d}x = \ensuremath{\int}_a^b (x-a)' f(x)  {\rm d}x = [(x-a) f(x)]_a^b - \ensuremath{\int}_a^b (x-a) f'(x) {\rm d} x \ccr
= (b-a) f(b) + \underbrace{\left(-\ensuremath{\int}_a^b (x-a) f'(x) {\rm d} x \right)}_\ensuremath{\delta}.
}
Recall that we can bound the absolute value of an integral by the supremum of the integrand times the width of the integration interval:
\[
\abs{\ensuremath{\int}_a^b g(x) {\rm d} x} \ensuremath{\leq} (b-a) \sup_{a \ensuremath{\leq} x \ensuremath{\leq} b}|g(x)|.
\]
The lemma thus follows since
\begin{align*}
\abs{\ensuremath{\int}_a^b (x-a) f'(x) {\rm d} x} &\ensuremath{\leq} (b-a) \sup_{a \ensuremath{\leq} x \ensuremath{\leq} b}|(x-a) f'(x)|  \\
&\ensuremath{\leq} (b-a) \sup_{a \ensuremath{\leq} x \ensuremath{\leq} b}|x-a| \sup_{a \ensuremath{\leq} x \ensuremath{\leq} b}|f'(x)|\\
&\ensuremath{\leq} M (b-a)^2.
\end{align*}
\ensuremath{\QED}

Now summing up the errors in each panel gives us the error of using the Rectangular rule:

\begin{theorem}[Rectangular Rule error] Assuming $f$ is differentiable we have
\[
\ensuremath{\int}_a^b f(x) {\rm d}x =  h \ensuremath{\sum}_{j=1}^n f(x_j) +  \ensuremath{\delta}
\]
where $|\ensuremath{\delta}| \ensuremath{\leq} M (b-a) h$ for $M = \sup_{a \ensuremath{\leq} x \ensuremath{\leq} b}|f'(x)|$, $h = (b-a)/n$ and $x_j = a + jh$.

\end{theorem}
\textbf{Proof} We split the integral into a sum of smaller integrals:
\[
\ensuremath{\int}_a^b f(x) {\rm d}x = \ensuremath{\sum}_{j=1}^n  \ensuremath{\int}_{x_{j-1}}^{x_j} f(x) {\rm d}x =
\ensuremath{\sum}_{j=1}^n  \br[(x_j - x_{j-1}) f(x_j) + \ensuremath{\delta}_j] =  h \ensuremath{\sum}_{j=1}^n f(x_j) +  \underbrace{\ensuremath{\sum}_{j=1}^n \ensuremath{\delta}_j}_\ensuremath{\delta}
\]
where $\ensuremath{\delta}_j$, the error on each panel as in the preceding lemma, satisfies
\[
|\ensuremath{\delta}_j| \ensuremath{\leq} (x_j-x_{j-1})^2 \sup_{x_{j-1} \ensuremath{\leq} x \ensuremath{\leq} x_j}|f'(x)| \ensuremath{\leq} M h^2.
\]
Thus using the triangular inequality we have
\[
|\ensuremath{\delta}| = \abs{ \ensuremath{\sum}_{j=1}^n \ensuremath{\delta}_j} \ensuremath{\leq} \ensuremath{\sum}_{j=1}^n |\ensuremath{\delta}_j| \ensuremath{\leq} M n h^2 = M(b-a)h.
\]
\ensuremath{\QED}

Note a consequence of this lemma is that the approximation converges as $n \ensuremath{\rightarrow} \ensuremath{\infty}$ (i.e. $h \ensuremath{\rightarrow} 0$). In the labs and problem sheets we will consider the left-sided rule:
\[
\ensuremath{\int}_a^b f(x) {\rm d}x \ensuremath{\approx}  h \ensuremath{\sum}_{j=0}^{n-1} f(x_j).
\]
We also consider the \emph{Trapezium rule}. Here we approximate an integral  by an affine function:
\[
\ensuremath{\int}_a^b f(x) {\rm d} x \ensuremath{\approx} \ensuremath{\int}_a^b {(b-x)f(a) + (x-a)f(b) \over b-a} \dx
= {b-a \over 2} \br[f(a) + f(b)].
\]
Subdividing an interval $a = x_0 < x_1 < \ensuremath{\ldots} < x_n = b$ and applying this approximation separately on each subinterval $[x_{j-1},x_j]$, where $h = (b-a)/n$ and $x_j = a + jh$, leads to the approximation
\[
\ensuremath{\int}_a^b f(x) {\rm d}x \ensuremath{\approx}  {h \over 2} f(a) + h \ensuremath{\sum}_{j=1}^{n-1} f(x_j) + {h \over 2} f(b)
\]
We shall see both experimentally and provably that this approximation converges faster than the rectangular rule.





\section{Divided Differences}
Given a function, how can we approximate its derivative at a point? We consider an intuitive approach to this problem using \emph{(Right-sided) Divided Differences}: 
\[
f'(x) \ensuremath{\approx} {f(x+h) - f(x) \over h}
\]
Note by the definition of the derivative we know that this approximation will converge to the true derivative as $h \ensuremath{\rightarrow} 0$. But in numerical approimxations we also need to consider the rate of convergence. 

Now in the previous section I mentioned there are three basic tools in analysis:  (1) integration-by-parts, (2) geometric series or (3) Taylor series. In this case we use (3):

\begin{proposition}[divided differences error] Suppose that $f$ is twice differrentable on the interval $(x,x+h)$. The error in approximating the derivative using divided differences is
\[
f'(x) = {f(x+h) - f(x) \over h} + \ensuremath{\delta}
\]
where $|\ensuremath{\delta}| \ensuremath{\leq} Mh/2$ for  $M = \sup_{x \ensuremath{\leq} t \ensuremath{\leq} x+h} |f''(t)|$.

\end{proposition}
\textbf{Proof} Follows immediately from Taylor's theorem:
\[
f(x+h) = f(x) + f'(x) h + \underbrace{{f''(t) \over 2} h^2}_{h \ensuremath{\delta}}
\]
for some $x \ensuremath{\leq} t \ensuremath{\leq} x+h$, by bounding:
\[
|\ensuremath{\delta}| \ensuremath{\leq} \abs{{f''(t) \over 2} h} \ensuremath{\leq} {M  h \over 2}.
\]
\ensuremath{\QED}

Unlike the rectangular rule, the computational cost of computing the divided difference is independent of $h$! We only need to evaluate a function $f$ twice and do a single division. Here we are assuming that the computational cost of evaluating $f$ is independent of the point of evaluation. Later we will investigate the details of how computers work with numbers via floating point,  and confirm that this is a sensible assumption.

So why not just set $h$ ridiculously small? In the lab we explore this question and observe that there are significant errors introduced in the numerical realisation of this algorithm. We will return to the question of understanding these errors after learning floating point numbers. 

There are alternative versions of divided differences. Left-side divided differences evaluates to the left of the point where wish to know the derivative:
\[
f'(x) \ensuremath{\approx} {f(x) - f(x-h) \over h}
\]
and central differences:
\[
f'(x) \ensuremath{\approx} {f(x + h) - f(x - h) \over 2h}
\]
We can further arrive at an approximation to the second derivative by composing a left- and right-sided finite difference:
\[
f''(x) \ensuremath{\approx} {f'(x+h) - f'(x) \over h} \ensuremath{\approx} {{f(x+h) - f(x) \over h} - {f(x) - f(x-h) \over h} \over h}
= {f(x+h) - 2f(x)  + f(x-h) \over h^2}
\]
In the lab we investigate the convergence rate of these approximations (in particular, that  central differences is more accurate than standard divided differences) and observe that they too suffer from unexplained (for now) loss of accuracy as $h \ensuremath{\rightarrow} 0$. In the problem sheet we prove the theoretical converge rate, which is never realised because of these errors.





\section{Dual Numbers}
In this chapter we introduce a mathematically beautiful  alternative to divided differences for computing derivatives: \emph{dual numbers}. These are a commutative ring that \emph{exactly} compute derivatives, which when implemented on a computer gives very high-accuracy approximations to derivatives. They underpin forward-mode \href{https://en.wikipedia.org/wiki/Automatic_differentiation}{automatic differentation}. Automatic differentiation  is a basic tool in Machine Learning for computing gradients necessary for training neural networks.

\begin{definition}[Dual numbers] Dual numbers $\ensuremath{\bbD}$ are a commutative ring (over $\ensuremath{\bbR}$) generated by $1$ and $\ensuremath{\epsilon}$ such that $\ensuremath{\epsilon}^2 = 0$. Dual numbers are typically written as $a + b \ensuremath{\epsilon}$ where $a$ and $b$ are real. \end{definition}

This is very much analoguous to complex numbers, which are a field generated by $1$ and $\I$ such that $\I^2 = -1$. Compare multiplication of each number type:
\meeq{
(a + b \I) (c + d \I) = ac + (bc + ad) \I + bd \I^2 = ac -bd + (bc + ad) \I \ccr
(a + b \ensuremath{\epsilon}) (c + d \ensuremath{\epsilon}) = ac + (bc + ad) \ensuremath{\epsilon} + bd \ensuremath{\epsilon}^2 = ac  + (bc + ad) \ensuremath{\epsilon} 
}
And just as we view $\ensuremath{\bbR} \ensuremath{\subset} \ensuremath{\bbC}$ by equating $a \ensuremath{\in} \ensuremath{\bbR}$ with $a + 0\I \ensuremath{\in} \ensuremath{\bbC}$, we can view $\ensuremath{\bbR} \ensuremath{\subset} \ensuremath{\bbD}$ by equating $a \ensuremath{\in} \ensuremath{\bbR}$ with $a + 0{\rm \ensuremath{\epsilon}} \ensuremath{\in} \ensuremath{\bbD}$.

\subsection{Differentiating polynomials}
Polynomials evaluated on dual numbers are well-defined as they depend only on the operations $+$ and $*$. From the formula for multiplication of dual numbers we deduce that evaluating a polynomial at a dual number $a + b \ensuremath{\epsilon}$ tells us the derivative of the polynomial at $a$:

\begin{theorem}[polynomials on dual numbers] Suppose $p$ is a polynomial. Then
\[
p(a + b \ensuremath{\epsilon}) = p(a) + b p'(a) \ensuremath{\epsilon}
\]
\end{theorem}
\textbf{Proof}

First consider $p(x) = x^n$ for $n \ensuremath{\geq} 0$.  The cases $n = 0$ and $n = 1$ are immediate. For $n > 1$ we have by induction:
\[
(a + b \ensuremath{\epsilon})^n = (a + b \ensuremath{\epsilon}) (a + b \ensuremath{\epsilon})^{n-1} = (a + b \ensuremath{\epsilon}) (a^{n-1} + (n-1) b a^{n-2} \ensuremath{\epsilon}) = a^n + b n a^{n-1} \ensuremath{\epsilon}.
\]
For a more general polynomial
\[
p(x) = \ensuremath{\sum}_{k=0}^n c_k x^k
\]
the result follows from linearity:
\[
p(a + b \ensuremath{\varepsilon}) = \ensuremath{\sum}_{k=0}^n c_k (a+b\ensuremath{\epsilon})^k = c_0 + \ensuremath{\sum}_{k=1}^n c_k (a^k +k b a^{k-1}\ensuremath{\epsilon})
= \ensuremath{\sum}_{k=0}^n c_k a^k + b \ensuremath{\sum}_{k=1}^n c_k k a^{k-1}\ensuremath{\epsilon} = p(a) + b p'(a) \ensuremath{\epsilon}.
\]
\ensuremath{\QED}

\begin{example}[differentiating polynomial] Consider computing $p'(2)$ where
\[
p(x) = (x-1)(x-2) + x^2.
\]
We can use dual numbers to differentiate, avoiding expanding in monomials or applying rules of differentiating:
\[
p(2+\ensuremath{\epsilon}) = (1+\ensuremath{\epsilon})\ensuremath{\epsilon} + (2+\ensuremath{\epsilon})^2 = \ensuremath{\epsilon} + 4 + 4\ensuremath{\epsilon} = 4 + \underbrace{5}_{p'(2)}\ensuremath{\epsilon}
\]
\end{example}

\subsection{Differentiating other functions}
We can extend real-valued differentiable functions to dual numbers in a similar manner. First, consider a standard function with a Taylor series (e.g. ${\rm cos}$, ${\rm sin}$, ${\rm exp}$, etc.)
\[
f(x) = \ensuremath{\sum}_{k=0}^\ensuremath{\infty} f_k x^k
\]
so that $a$ is inside the radius of convergence. This leads naturally to a definition on dual numbers:
\meeq{
f(a + b \ensuremath{\epsilon}) = \ensuremath{\sum}_{k=0}^\ensuremath{\infty} f_k (a + b \ensuremath{\epsilon})^k = f_0 + \ensuremath{\sum}_{k=1}^\ensuremath{\infty} f_k (a^k + k a^{k-1} b \ensuremath{\epsilon}) = \ensuremath{\sum}_{k=0}^\ensuremath{\infty} f_k a^k +  \ensuremath{\sum}_{k=1}^\ensuremath{\infty} f_k k a^{k-1} b \ensuremath{\epsilon}  \ccr
  = f(a) + b f'(a) \ensuremath{\epsilon}
}
More generally, given a differentiable function we can extend it to dual numbers:

\begin{definition}[dual extension] Suppose a real-valued function $f$ is differentiable at $a$. If
\[
f(a + b \ensuremath{\epsilon}) = f(a) + b f'(a) \ensuremath{\epsilon}
\]
then we say that it is a \emph{dual extension at} $a$.

Thus, for basic functions we have natural extensions:


\begin{align*}
\exp(a + b \ensuremath{\epsilon}) &:= \exp(a) + b \exp(a) \ensuremath{\epsilon} \\
\sin(a + b \ensuremath{\epsilon}) &:= \sin(a) + b \cos(a) \ensuremath{\epsilon} \\
\cos(a + b \ensuremath{\epsilon}) &:= \cos(a) - b \sin(a) \ensuremath{\epsilon} \\
\log(a + b \ensuremath{\epsilon}) &:= \log(a) + {b \over a} \ensuremath{\epsilon} \\
\sqrt{a+b \ensuremath{\epsilon}} &:= \sqrt{a} + {b \over 2 \sqrt{a}} \ensuremath{\epsilon} \\
|a + b \ensuremath{\epsilon}| &:= |a| + b\, {\rm sign} a\, \ensuremath{\epsilon}
\end{align*}
provided the function is differentiable at $a$. Note the last example does not have a convergent Taylor series (at 0) but we can still extend it where it is differentiable.

Going further, we can add, multiply, and compose such functions:

\begin{lemma}[product and chain rule] If $f$ is a dual extension at $g(a)$ and $g$ is a dual extension at $a$, then $q(x) := f(g(x))$ is a dual extension at $a$. If $f$ and $g$ are dual extensions at $a$ then  $r(x) := f(x) g(x)$ is also dual extensions at $a$. In other words:
\meeq{
q(a+b \ensuremath{\epsilon}) = q(a) + b q'(a) \ensuremath{\epsilon} \ccr
r(a+b \ensuremath{\epsilon}) = r(a) + b r'(a) \ensuremath{\epsilon}
}
\end{lemma}
\textbf{Proof} For $q$ it follows immediately:
\meeq{
q(a + b \ensuremath{\epsilon}) = f(g(a + b \ensuremath{\epsilon})) = f(g(a) + b g'(a) \ensuremath{\epsilon}) \ccr
= f(g(a)) + b g'(a) f'(g(a))\ensuremath{\epsilon} = q(a) + b q'(a) \ensuremath{\epsilon}.
}
For $r$ we have
\meeq{
r(a + b \ensuremath{\epsilon}) = f(a+b \ensuremath{\epsilon} )g(a+b \ensuremath{\epsilon} )= (f(a) + b f'(a) \ensuremath{\epsilon})(g(a) + b g'(a) \ensuremath{\epsilon}) \ccr
= f(a)g(a) + b (f'(a)g(a) + f(a)g'(a)) \ensuremath{\epsilon} = r(a) +b r'(a) \ensuremath{\epsilon}.
}
\end{definition}

A simple corollary is that any function defined in terms of addition, multiplication, composition, etc. of functions that are dual with differentiation will be differentiable via dual numbers.

\begin{example}[differentiating non-polynomial]

Consider differentiating $f(x) =  \exp(x^2 + \E^x)$ at the point $a = 1$ by evaluating on the duals:
\[
f(1 + \ensuremath{\epsilon}) = \exp(1 + 2\ensuremath{\epsilon} + \E + \E \ensuremath{\epsilon}) =  \exp(1 + \E) + \exp(1 + \E) (2 + \E) \ensuremath{\epsilon}.
\]
Therefore we deduce that
\[
f'(1) = \exp(1 + \E) (2 + \E).
\]
\end{example}






\end{document}