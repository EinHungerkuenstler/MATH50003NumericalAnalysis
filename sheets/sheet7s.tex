\documentclass[12pt,a4paper]{article}

\usepackage[a4paper,text={16.5cm,25.2cm},centering]{geometry}
\usepackage{lmodern}
\usepackage{amssymb,amsmath}
\usepackage{bm}
\usepackage{graphicx}
\usepackage{microtype}
\usepackage{hyperref}
\usepackage[usenames,dvipsnames]{xcolor}
\setlength{\parindent}{0pt}
\setlength{\parskip}{1.2ex}




\hypersetup
       {   pdfauthor = {  },
           pdftitle={  },
           colorlinks=TRUE,
           linkcolor=black,
           citecolor=blue,
           urlcolor=blue
       }




\usepackage{upquote}
\usepackage{listings}
\usepackage{xcolor}
\lstset{
    basicstyle=\ttfamily\footnotesize,
    upquote=true,
    breaklines=true,
    breakindent=0pt,
    keepspaces=true,
    showspaces=false,
    columns=fullflexible,
    showtabs=false,
    showstringspaces=false,
    escapeinside={(*@}{@*)},
    extendedchars=true,
}
\newcommand{\HLJLt}[1]{#1}
\newcommand{\HLJLw}[1]{#1}
\newcommand{\HLJLe}[1]{#1}
\newcommand{\HLJLeB}[1]{#1}
\newcommand{\HLJLo}[1]{#1}
\newcommand{\HLJLk}[1]{\textcolor[RGB]{148,91,176}{\textbf{#1}}}
\newcommand{\HLJLkc}[1]{\textcolor[RGB]{59,151,46}{\textit{#1}}}
\newcommand{\HLJLkd}[1]{\textcolor[RGB]{214,102,97}{\textit{#1}}}
\newcommand{\HLJLkn}[1]{\textcolor[RGB]{148,91,176}{\textbf{#1}}}
\newcommand{\HLJLkp}[1]{\textcolor[RGB]{148,91,176}{\textbf{#1}}}
\newcommand{\HLJLkr}[1]{\textcolor[RGB]{148,91,176}{\textbf{#1}}}
\newcommand{\HLJLkt}[1]{\textcolor[RGB]{148,91,176}{\textbf{#1}}}
\newcommand{\HLJLn}[1]{#1}
\newcommand{\HLJLna}[1]{#1}
\newcommand{\HLJLnb}[1]{#1}
\newcommand{\HLJLnbp}[1]{#1}
\newcommand{\HLJLnc}[1]{#1}
\newcommand{\HLJLncB}[1]{#1}
\newcommand{\HLJLnd}[1]{\textcolor[RGB]{214,102,97}{#1}}
\newcommand{\HLJLne}[1]{#1}
\newcommand{\HLJLneB}[1]{#1}
\newcommand{\HLJLnf}[1]{\textcolor[RGB]{66,102,213}{#1}}
\newcommand{\HLJLnfm}[1]{\textcolor[RGB]{66,102,213}{#1}}
\newcommand{\HLJLnp}[1]{#1}
\newcommand{\HLJLnl}[1]{#1}
\newcommand{\HLJLnn}[1]{#1}
\newcommand{\HLJLno}[1]{#1}
\newcommand{\HLJLnt}[1]{#1}
\newcommand{\HLJLnv}[1]{#1}
\newcommand{\HLJLnvc}[1]{#1}
\newcommand{\HLJLnvg}[1]{#1}
\newcommand{\HLJLnvi}[1]{#1}
\newcommand{\HLJLnvm}[1]{#1}
\newcommand{\HLJLl}[1]{#1}
\newcommand{\HLJLld}[1]{\textcolor[RGB]{148,91,176}{\textit{#1}}}
\newcommand{\HLJLs}[1]{\textcolor[RGB]{201,61,57}{#1}}
\newcommand{\HLJLsa}[1]{\textcolor[RGB]{201,61,57}{#1}}
\newcommand{\HLJLsb}[1]{\textcolor[RGB]{201,61,57}{#1}}
\newcommand{\HLJLsc}[1]{\textcolor[RGB]{201,61,57}{#1}}
\newcommand{\HLJLsd}[1]{\textcolor[RGB]{201,61,57}{#1}}
\newcommand{\HLJLsdB}[1]{\textcolor[RGB]{201,61,57}{#1}}
\newcommand{\HLJLsdC}[1]{\textcolor[RGB]{201,61,57}{#1}}
\newcommand{\HLJLse}[1]{\textcolor[RGB]{59,151,46}{#1}}
\newcommand{\HLJLsh}[1]{\textcolor[RGB]{201,61,57}{#1}}
\newcommand{\HLJLsi}[1]{#1}
\newcommand{\HLJLso}[1]{\textcolor[RGB]{201,61,57}{#1}}
\newcommand{\HLJLsr}[1]{\textcolor[RGB]{201,61,57}{#1}}
\newcommand{\HLJLss}[1]{\textcolor[RGB]{201,61,57}{#1}}
\newcommand{\HLJLssB}[1]{\textcolor[RGB]{201,61,57}{#1}}
\newcommand{\HLJLnB}[1]{\textcolor[RGB]{59,151,46}{#1}}
\newcommand{\HLJLnbB}[1]{\textcolor[RGB]{59,151,46}{#1}}
\newcommand{\HLJLnfB}[1]{\textcolor[RGB]{59,151,46}{#1}}
\newcommand{\HLJLnh}[1]{\textcolor[RGB]{59,151,46}{#1}}
\newcommand{\HLJLni}[1]{\textcolor[RGB]{59,151,46}{#1}}
\newcommand{\HLJLnil}[1]{\textcolor[RGB]{59,151,46}{#1}}
\newcommand{\HLJLnoB}[1]{\textcolor[RGB]{59,151,46}{#1}}
\newcommand{\HLJLoB}[1]{\textcolor[RGB]{102,102,102}{\textbf{#1}}}
\newcommand{\HLJLow}[1]{\textcolor[RGB]{102,102,102}{\textbf{#1}}}
\newcommand{\HLJLp}[1]{#1}
\newcommand{\HLJLc}[1]{\textcolor[RGB]{153,153,119}{\textit{#1}}}
\newcommand{\HLJLch}[1]{\textcolor[RGB]{153,153,119}{\textit{#1}}}
\newcommand{\HLJLcm}[1]{\textcolor[RGB]{153,153,119}{\textit{#1}}}
\newcommand{\HLJLcp}[1]{\textcolor[RGB]{153,153,119}{\textit{#1}}}
\newcommand{\HLJLcpB}[1]{\textcolor[RGB]{153,153,119}{\textit{#1}}}
\newcommand{\HLJLcs}[1]{\textcolor[RGB]{153,153,119}{\textit{#1}}}
\newcommand{\HLJLcsB}[1]{\textcolor[RGB]{153,153,119}{\textit{#1}}}
\newcommand{\HLJLg}[1]{#1}
\newcommand{\HLJLgd}[1]{#1}
\newcommand{\HLJLge}[1]{#1}
\newcommand{\HLJLgeB}[1]{#1}
\newcommand{\HLJLgh}[1]{#1}
\newcommand{\HLJLgi}[1]{#1}
\newcommand{\HLJLgo}[1]{#1}
\newcommand{\HLJLgp}[1]{#1}
\newcommand{\HLJLgs}[1]{#1}
\newcommand{\HLJLgsB}[1]{#1}
\newcommand{\HLJLgt}[1]{#1}


\def\endash{–}
\def\bbD{ {\mathbb D} }
\def\bbZ{ {\mathbb Z} }
\def\bbR{ {\mathbb R} }
\def\bbC{ {\mathbb C} }

\def\x{ {\vc x} }
\def\a{ {\vc a} }
\def\b{ {\vc b} }
\def\e{ {\vc e} }
\def\f{ {\vc f} }
\def\u{ {\vc u} }
\def\v{ {\vc v} }
\def\y{ {\vc y} }
\def\z{ {\vc z} }
\def\w{ {\vc w} }

\def\bt{ {\tilde b} }
\def\ct{ {\tilde c} }
\def\Ut{ {\tilde U} }
\def\Qt{ {\tilde Q} }
\def\Rt{ {\tilde R} }
\def\Xt{ {\tilde X} }
\def\acos{ {\rm acos}\, }

\def\red#1{ {\color{red} #1} }
\def\blue#1{ {\color{blue} #1} }
\def\green#1{ {\color{ForestGreen} #1} }
\def\magenta#1{ {\color{magenta} #1} }


\def\addtab#1={#1\;&=}

\def\meeq#1{\def\ccr{\\\addtab}
%\tabskip=\@centering
 \begin{align*}
 \addtab#1
 \end{align*}
  }  
  
  \def\leqaddtab#1\leq{#1\;&\leq}
  \def\mleeq#1{\def\ccr{\\\addtab}
%\tabskip=\@centering
 \begin{align*}
 \leqaddtab#1
 \end{align*}
  }  


\def\vc#1{\mbox{\boldmath$#1$\unboldmath}}

\def\vcsmall#1{\mbox{\boldmath$\scriptstyle #1$\unboldmath}}

\def\vczero{{\mathbf 0}}


%\def\beginlist{\begin{itemize}}
%
%\def\endlist{\end{itemize}}


\def\pr(#1){\left({#1}\right)}
\def\br[#1]{\left[{#1}\right]}
\def\fbr[#1]{\!\left[{#1}\right]}
\def\set#1{\left\{{#1}\right\}}
\def\ip<#1>{\left\langle{#1}\right\rangle}
\def\iip<#1>{\left\langle\!\langle{#1}\right\rangle\!\rangle}

\def\norm#1{\left\| #1 \right\|}

\def\abs#1{\left|{#1}\right|}
\def\fpr(#1){\!\pr({#1})}

\def\Re{{\rm Re}\,}
\def\Im{{\rm Im}\,}

\def\floor#1{\left\lfloor#1\right\rfloor}
\def\ceil#1{\left\lceil#1\right\rceil}


\def\mapengine#1,#2.{\mapfunction{#1}\ifx\void#2\else\mapengine #2.\fi }

\def\map[#1]{\mapengine #1,\void.}

\def\mapenginesep_#1#2,#3.{\mapfunction{#2}\ifx\void#3\else#1\mapengine #3.\fi }

\def\mapsep_#1[#2]{\mapenginesep_{#1}#2,\void.}


\def\vcbr{\br}


\def\bvect[#1,#2]{
{
\def\dots{\cdots}
\def\mapfunction##1{\ | \  ##1}
\begin{pmatrix}
		 \,#1\map[#2]\,
\end{pmatrix}
}
}

\def\vect[#1]{
{\def\dots{\ldots}
	\vcbr[{#1}]
}}

\def\vectt[#1]{
{\def\dots{\ldots}
	\vect[{#1}]^{\top}
}}

\def\Vectt[#1]{
{
\def\mapfunction##1{##1 \cr} 
\def\dots{\vdots}
	\begin{pmatrix}
		\map[#1]
	\end{pmatrix}
}}



\def\thetaB{\mbox{\boldmath$\theta$}}
\def\zetaB{\mbox{\boldmath$\zeta$}}


\def\newterm#1{{\it #1}\index{#1}}


\def\TT{{\mathbb T}}
\def\C{{\mathbb C}}
\def\R{{\mathbb R}}
\def\II{{\mathbb I}}
\def\F{{\mathcal F}}
\def\E{{\rm e}}
\def\I{{\rm i}}
\def\D{{\rm d}}
\def\dx{\D x}
\def\ds{\D s}
\def\dt{\D t}
\def\CC{{\cal C}}
\def\DD{{\cal D}}
\def\U{{\mathbb U}}
\def\A{{\cal A}}
\def\K{{\cal K}}
\def\DTU{{\cal D}_{{\rm T} \rightarrow {\rm U}}}
\def\LL{{\cal L}}
\def\B{{\cal B}}
\def\T{{\cal T}}
\def\W{{\cal W}}


\def\tF_#1{{\tt F}_{#1}}
\def\Fm{\tF_m}
\def\Fab{\tF_{\alpha,\beta}}
\def\FC{\T}
\def\FCpmz{\FC^{\pm {\rm z}}}
\def\FCz{\FC^{\rm z}}

\def\tFC_#1{{\tt T}_{#1}}
\def\FCn{\tFC_n}

\def\rmz{{\rm z}}

\def\chapref#1{Chapter~\ref{Chapter:#1}}
\def\secref#1{Section~\ref{Section:#1}}
\def\exref#1{Exercise~\ref{Exercise:#1}}
\def\lmref#1{Lemma~\ref{Lemma:#1}}
\def\propref#1{Proposition~\ref{Proposition:#1}}
\def\warnref#1{Warning~\ref{Warning:#1}}
\def\thref#1{Theorem~\ref{Theorem:#1}}
\def\defref#1{Definition~\ref{Definition:#1}}
\def\probref#1{Problem~\ref{Problem:#1}}
\def\corref#1{Corollary~\ref{Corollary:#1}}

\def\sgn{{\rm sgn}\,}
\def\Ai{{\rm Ai}\,}
\def\Bi{{\rm Bi}\,}
\def\wind{{\rm wind}\,}
\def\erf{{\rm erf}\,}
\def\erfc{{\rm erfc}\,}
\def\qqquad{\qquad\quad}
\def\qqqquad{\qquad\qquad}


\def\spand{\hbox{ and }}
\def\spodd{\hbox{ odd}}
\def\speven{\hbox{ even}}
\def\qand{\quad\hbox{and}\quad}
\def\qqand{\qquad\hbox{and}\qquad}
\def\qfor{\quad\hbox{for}\quad}
\def\qqfor{\qquad\hbox{for}\qquad}
\def\qas{\quad\hbox{as}\quad}
\def\qqas{\qquad\hbox{as}\qquad}
\def\qor{\quad\hbox{or}\quad}
\def\qqor{\qquad\hbox{or}\qquad}
\def\qqwhere{\qquad\hbox{where}\qquad}



%%% Words

\def\naive{na\"\i ve\xspace}
\def\Jmap{Joukowsky map\xspace}
\def\Mobius{M\"obius\xspace}
\def\Holder{H\"older\xspace}
\def\Mathematica{{\sc Mathematica}\xspace}
\def\apriori{apriori\xspace}
\def\WHf{Weiner--Hopf factorization\xspace}
\def\WHfs{Weiner--Hopf factorizations\xspace}

\def\Jup{J_\uparrow^{-1}}
\def\Jdown{J_\downarrow^{-1}}
\def\Jin{J_+^{-1}}
\def\Jout{J_-^{-1}}



\def\bD{\D\!\!\!^-}




\def\questionequals{= \!\!\!\!\!\!{\scriptstyle ? \atop }\,\,\,}

\def\elll#1{\ell^{\lambda,#1}}
\def\elllp{\ell^{\lambda,p}}
\def\elllRp{\ell^{(\lambda,R),p}}


\def\elllRpz_#1{\ell_{#1{\rm z}}^{(\lambda,R),p}}


\def\sopmatrix#1{\begin{pmatrix}#1\end{pmatrix}}


\def\bbR{{\mathbb R}}
\def\bbC{{\mathbb C}}


\begin{document}



\textbf{Numerical Analysis MATH50003 (2023\ensuremath{\endash}24) Problem Sheet 7}

\textbf{Problem 1(a)} Show for a unitary matrix $Q \ensuremath{\in} U(n)$ and a vector $\ensuremath{\bm{\x}} \ensuremath{\in} \ensuremath{\bbC}^n$ that multiplication by $Q$ preserve the 2-norm: $\|Q \ensuremath{\bm{\x}}\| = \|\ensuremath{\bm{\x}}\|.$

\textbf{SOLUTION}
\[
\|Q \ensuremath{\bm{\x}}\|^2 = (Q \ensuremath{\bm{\x}})^\ensuremath{\star} Q \ensuremath{\bm{\x}} = \ensuremath{\bm{\x}}^\ensuremath{\star} Q^\ensuremath{\star} Q \ensuremath{\bm{\x}} = \ensuremath{\bm{\x}}^\ensuremath{\star}  \ensuremath{\bm{\x}} = \|\ensuremath{\bm{\x}}\|^2
\]
\textbf{END}

\textbf{Problem 1(b)} Show that the eigenvalues $\ensuremath{\lambda}$ of a unitary matrix $Q$ are on the unit circle: $|\ensuremath{\lambda}| = 1$. Hint: recall for any eigenvalue $\ensuremath{\lambda}$ that there exists a unit eigenvector $\ensuremath{\bm{\v}} \ensuremath{\in} \ensuremath{\bbC}^n$ (satisfying $\| \ensuremath{\bm{\v}} \| = 1$). 

\textbf{SOLUTION} Let $\ensuremath{\bm{\v}}$ be a unit eigenvector corresponding to $\ensuremath{\lambda}$: $Q \ensuremath{\bm{\v}} = \ensuremath{\lambda} \ensuremath{\bm{\v}}$ with $\|\ensuremath{\bm{\v}}\| = 1$. Then
\[
1 = \| \ensuremath{\bm{\v}} \| = \|Q \ensuremath{\bm{\v}} \| =  \| \ensuremath{\lambda} \ensuremath{\bm{\v}} \| = |\ensuremath{\lambda}|.
\]
\textbf{END}

\textbf{Problem 1(c)} Show for an orthogonal matrix $Q \ensuremath{\in} O(n)$ that $\det Q = \ensuremath{\pm}1$. Give an example of $Q \ensuremath{\in} U(n)$ such that $\det Q \ensuremath{\neq} \ensuremath{\pm}1$. Hint: recall for any real matrices $A$ and $B$ that $\det A = \det A^\ensuremath{\top}$ and $\det(AB) = \det A \det B$.

\textbf{SOLUTION}
\[
(\det Q)^2 = (\det Q^\ensuremath{\top})(\det Q) = \det Q^\ensuremath{\top}Q = \det I = 1.
\]
An example would be a 1 \ensuremath{\times} 1 complex-valued matrix $\exp({\rm i})$.

\textbf{END}

\textbf{Problem 1(d)} A normal matrix commutes with its adjoint. Show that $Q \ensuremath{\in} U(n)$ is normal.

\textbf{SOLUTION}
\[
 QQ^\ensuremath{\star} = I = Q^\ensuremath{\star}Q
\]
\textbf{END}

\textbf{Problem 1(e)}  The spectral theorem states that any normal matrix is unitarily diagonalisable: if $A$ is normal then $A = V \ensuremath{\Lambda} V^\ensuremath{\star}$ where $V \ensuremath{\in} U(n)$ and $\ensuremath{\Lambda}$ is diagonal. Use this to show that  $Q \ensuremath{\in} U(n)$ is equal to $I$ if and only if all its eigenvalues are 1.

\textbf{SOLUTION}

Note that $Q$ is normal and therefore by the spectral theorem for  normal matrices we have
\[
Q = V \ensuremath{\Lambda} V^\ensuremath{\star} = V V^\ensuremath{\star} = I
\]
since $V$ is unitary. 

\textbf{END}

\textbf{Problem 2} Consider the vectors
\[
\ensuremath{\bm{\a}} = \begin{bmatrix} 1 \\ 2 \\ 2 \end{bmatrix}\qquad\hbox{and}\qquad  \ensuremath{\bm{\b}} = \begin{bmatrix} 1 \\ 2{\rm i} \\ 2 \end{bmatrix}.
\]
Use reflections to determine the entries of orthogonal/unitary matrices $Q_1, Q_2, Q_3$ such that
\[
Q_1 \ensuremath{\bm{\a}} = \begin{bmatrix} 3 \\ 0 \\ 0 \end{bmatrix}, Q_2 \ensuremath{\bm{\a}} = \begin{bmatrix} -3 \\ 0 \\ 0 \end{bmatrix},
Q_3 \ensuremath{\bm{\b}} = \begin{bmatrix} -3 \\ 0 \\ 0 \end{bmatrix}
\]
\textbf{SOLUTION}

For $Q_1$: we have
\begin{align*}
\ensuremath{\bm{\y}} &= \ensuremath{\bm{\a}} - \| \ensuremath{\bm{\a}}\| \ensuremath{\bm{\e}}_1 =  \begin{bmatrix} -2 \\ 2 \\ 2 \end{bmatrix} \\
\ensuremath{\bm{\w}} &= {\ensuremath{\bm{\y}} \over \|\ensuremath{\bm{\y}}\|} = {1 \over \sqrt{3}} \begin{bmatrix} -1 \\ 1 \\ 1 \end{bmatrix} \\
Q_1 &= Q_{\ensuremath{\bm{\w}}} = I - {2 \over 3} \begin{bmatrix} -1 \\ 1 \\ 1 \end{bmatrix}  [-1\ 1\ 1] = 
I - {2 \over 3} \begin{bmatrix} 1 & -1 & -1 \\ -1 & 1 & 1 \\ -1 & 1 & 1 \end{bmatrix} \\
&={1 \over 3} \begin{bmatrix} 1 & 2 & 2 \\ 2 & 1 & -2 \\ 2 & -2 & 1 \end{bmatrix} 
\end{align*}
For $Q_2$: we have
\begin{align*}
\ensuremath{\bm{\y}} &= \ensuremath{\bm{\a}} + \| \ensuremath{\bm{\a}}\| \ensuremath{\bm{\e}}_1 =  \begin{bmatrix} 4 \\ 2 \\ 2 \end{bmatrix} \\
\ensuremath{\bm{\w}} &= {\ensuremath{\bm{\y}} \over \|\ensuremath{\bm{\y}}\|} = {1 \over \sqrt{6}} \begin{bmatrix} 2 \\ 1 \\ 1 \end{bmatrix} \\
Q_2 &= Q_{\ensuremath{\bm{\w}}} = I - {1 \over 3} \begin{bmatrix} 2 \\ 1 \\ 1 \end{bmatrix}  [2\ 1\ 1] = 
I - {1 \over 3} \begin{bmatrix} 4 & 2 & 2 \\ 2 & 1  & 1 \\ 2&  1 & 1 \end{bmatrix} \\
&={1 \over 3} \begin{bmatrix} -1 & -2 & -2 \\ -2& 2 &-1 \\ -2& -1& 2 \end{bmatrix} 
\end{align*}
For $Q_3$ we just need to be careful to conjugate:
\begin{align*}
\ensuremath{\bm{\y}} &= \ensuremath{\bm{\b}} + \| \ensuremath{\bm{\b}}\| \ensuremath{\bm{\e}}_1 =  \begin{bmatrix} 4 \\ 2{\rm i} \\ 2 \end{bmatrix} \\
\ensuremath{\bm{\w}} &= {\ensuremath{\bm{\y}} \over \|\ensuremath{\bm{\y}}\|} = {1 \over \sqrt{6}} \begin{bmatrix}2 \\ {\rm i} \\ 1 \end{bmatrix} \\
Q_3 &= Q_{\ensuremath{\bm{\w}}} = I - {1 \over 3} \begin{bmatrix} 2 \\ {\rm i} \\ 1 \end{bmatrix}  [2\ -{\rm i}\ 1] = 
I - {1 \over 3} \begin{bmatrix} 4 & -2{\rm i} & 2 \\ 
                                2{\rm i}& 1 & {\rm i} \\ 
                                2 &-{\rm i} & 1 \end{bmatrix} \\
&={1 \over 3} \begin{bmatrix} -1 & 2{\rm i} & -2 \\ 
                            -2{\rm i} & 2 & -{\rm i} \\ 
                            -2& {\rm i} & 2 \end{bmatrix} 
\end{align*}
\textbf{END}

\textbf{Problem 3(a)} What simple rotation matrices $Q_1,Q_2 \ensuremath{\in} SO(2)$ have the property that:
\[
Q_1 \begin{bmatrix} 1 \\ 2 \end{bmatrix} =\begin{bmatrix} \sqrt{5} \\ 0 \end{bmatrix},  Q_2 \begin{bmatrix} \sqrt{5} \\ 2 \end{bmatrix} =  \begin{bmatrix} 3 \\ 0 \end{bmatrix}
\]
\textbf{SOLUTION}

The rotation that takes $[x,y]$ to the x-axis is
\[
{1 \over \sqrt{x^2+y^2}} \begin{bmatrix}
x & y \\
-y & x
\end{bmatrix}.
\]
Hence we get
\begin{align*}
Q_1 &= {1 \over \sqrt{5}} \begin{bmatrix} 1 & 2 \\ -2 & 1 \end{bmatrix} \\
Q_2 &= {1 \over 3} \begin{bmatrix} \sqrt{5} & 2 \\ -2 & \sqrt{5} \end{bmatrix}
\end{align*}
\textbf{END}

\textbf{Problem 3(b)} Find an orthogonal matrix that is a product of two simple rotations but acting on two different subspaces:
\[
Q  = \underbrace{\begin{bmatrix} \cos \ensuremath{\theta}_2 & & -\sin \ensuremath{\theta}_2  \\ & 1 \\
\sin \ensuremath{\theta}_2  & & \cos \ensuremath{\theta}_2  \end{bmatrix}}_{Q_2} \underbrace{\begin{bmatrix} \cos \ensuremath{\theta}_1 & -\sin \ensuremath{\theta}_1  \\ \sin \ensuremath{\theta}_1 & \cos \ensuremath{\theta}_1 \\ && 1 \end{bmatrix}}_{Q_1}
\]
so that, for $\ensuremath{\bm{\a}}$ defined above,
\[
Q \ensuremath{\bm{\a}}  = \begin{bmatrix} \|\ensuremath{\bm{\a}}\| \\ 0 \\ 0 \end{bmatrix}.
\]
Hint: you do not need to determine $\ensuremath{\theta}_1, \ensuremath{\theta}_2$, instead you can write the entries of  $Q_1, Q_2$  directly using just square-roots. 

\textbf{SOLUTION}

We use $Q_1$ to introduce a 0 in the second entry by rotating the vector $[1,2]$:
\[
Q_1 =  \begin{bmatrix} 1/\sqrt{5}  & 2/\sqrt{5} \\ -2/\sqrt{5} & 1/\sqrt{5} \\ && 1 \end{bmatrix}
\]
so that
\[
Q_1 \ensuremath{\bm{\a}} = \begin{bmatrix} \sqrt{5} \\ 0 \\ 2 \end{bmatrix}.
\]
Now we use the matrix that rotates the vector $[\sqrt{5},2]$ whose norm is $3$ to deduce the entries
\[
Q_2 = \begin{bmatrix} \sqrt{5}/3 & & 2/3  \\ & 1 \\
-2/3  & & \sqrt{5}/3  \end{bmatrix}
\]
so that
\[
Q_2 Q_1 = \begin{bmatrix} 1/3 & 2/3 & 2/3 \\ 
 -2/\sqrt{5} & 1/\sqrt{5} & \\ 
 -2/(3\sqrt{5}) & -4/(3\sqrt{5}) & \sqrt{5}/3
 \end{bmatrix}
\]
\textbf{END}

\textbf{Problem 4(a)} Show that every matrix $A \ensuremath{\in} \ensuremath{\bbR}^{m \ensuremath{\times} n}$ has a  QR factorisation such that the diagonal of $R$ is non-negative. Make sure to include the case of more columns than rows (i.e. $m < n$). 

\textbf{SOLUTION}

We first show for $m < n$ that a QR decomposition exists. Writing
\[
A = [\ensuremath{\bm{\a}}_1 | \ensuremath{\cdots} | \ensuremath{\bm{\a}}_n]
\]
and taking the first $m$ columns (so that it is square) we can write $[\ensuremath{\bm{\a}}_1 | \ensuremath{\cdots} | \ensuremath{\bm{\a}}_m] = Q R_m$. It follows that $R := Q^\ensuremath{\star} A$ is right-triangular.

We can write:
\[
D = \begin{bmatrix} \hbox{sign}(r_{11}) \\ & \ensuremath{\ddots}  \\ && \hbox{sign}(r_{pp}) \end{bmatrix}
\]
where $p = \min(m,n)$ and we define $\hbox{sign}(0) = 1$. Note that $D^\ensuremath{\top} D = I$. Thus we can write: $A = Q R = Q D D R$ where $(QD)$ is orthogonal and $DR$ is upper-triangular with positive entries.

\textbf{END}

\textbf{Problem 4(b)} Show that the QR factorisation of a square invertible matrix $A \ensuremath{\in} \ensuremath{\bbR}^{n \ensuremath{\times} n}$ is unique, provided that the diagonal of $R$ is positive.

\textbf{SOLUTION}

Assume there is a second factorisation also with positive diagonal
\[
A = QR = \Qt  \Rt
\]
Then we know
\[
Q^\ensuremath{\top} \Qt  = R \Rt^{-1}
\]
Note $Q^\ensuremath{\top} \Qt $ is a product of orthogonal matrices so is also orthogonal. It's eigenvalues are the same as $R \Rt^{-1}$, which is upper triangular. The eigenvalues of an upper triangular matrix are the diagonal entries, which in this case are all positive. Since all eigenvalues of an orthogonal matrix are on the unit circle (see Q1(b) above) we know all $m$ eigenvalues of $Q^\ensuremath{\top} \Qt$ are 1. By Q1(e) above, this means that $Q^\ensuremath{\top} \Qt  = I$. Hence
\[
\Qt = (Q^\ensuremath{\top})^{-1} = Q
\]
and
\[
\Rt = (\Qt)^{-1}A =  R.
\]
\textbf{END}



\end{document}